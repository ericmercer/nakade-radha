Parallel programming is being used increasingly in large-scale applications. Task parallelism is a branch of parallel programming in which different instructions are run on different sets of data simultaneously on the available processor cores. This gives rise to different memory access patterns. When two or more processes access a memory location such that at least one of the accesses is a  'write', a data-race is created in the program. Data-races cause the output of the program to be non-deterministic. A computation graph of a task parallel program can be used to represents the execution of the program in the form of a directed acyclic graph. In this research, we propose a method for detecting data-races with the help of a computation graph of the program. We also discuss ways to test the effectiveness of this method.
