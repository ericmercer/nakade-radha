\section{Introduction}

Until recently, the speed of the processors was expected to increase rapidly over time with sustained technology advances, and the motivation for parallel computing was low. But now, clock frequencies for individual processors are no longer increasing. The reason for that being the power consumed by a processor using current device technologies varies as the cube of the frequency. Processor chips in laptop and personal computers are typically limited to consume at most 75 Watts because a larger power consumption would lead to chips overheating and malfunctioning, given the limited air­cooling capacity of these computers. This power limitation has been referred to as the “Power Wall”. The Power Wall is even more critical in smart­phones and hand held devices because larger power consumption leads to a shorter battery life. Therefore, clock frequency can no longer be increased. When multiple cores are used in parallel, the speed of computation is increased but not the power consumption. This is the main motivation behind parallel programming.

 However, the introduction of concurrency in programs leads to non deterministic behavior. Many problems such as deadlocks, data races etc. arise when concurrent programs are not written correctly. Such issues are very hard to debug.
 
 Many tools have been developed to detect deadlocks, data races, non-determinism in parallel programs. These tools assist the developers of concurrent programs to detect erroneous behavior of their programs and rectify their implementations. However, in large systems it takes a great deal of effort to locate and rectify such errors. Especially in safety-critical systems, it is very important to ensure that the concurrent programs are free of such errors.
 
 The programmers are, therefore, trying to develop parallel programming languages that ensure safety against concurrency related errors. One such language is Habanero Java. Habanero Java claims to provide safety against deadlocks and non-determinism if the constructs in the language are used correctly. This can save lot of efforts of verification on the part of developers using this language to implement their applications. However, the claims made by Habanero Java Language have not been verified yet. The work described in this proposal focuses on the verification of safety properties of HJ language. The verification process consists of two steps. Firstly, creation of a computation graph of the program. Secondly, analysis of the graph to verify the properties. 
 
 A computation graph of a program is a directed acyclic graph(DAG) structure that captures the meaning of the program's execution as a partial order. A tool called HJ-Viz was developed at the Rice University to visualize the computation graphs of HJ programs. HJ-Viz processes the event logs produced by the HJLib runtime and generates a dot representation of the computation graph.The dot file can later be processed and displayed in the user's web browser using the HJ-Viz server. The drawback of this tool is that it does not store the computation graph in a logical data structure that can be traversed and analyzed  to verify the properties of the program. 
 
 Hence, we need to create a tool that can build a computation graph of an HJ program during runtime and store the information in a data-structure. We have chosen Java Path Finder (JPF) to create the improved version of the computation graph builder.  JPF takes java classs files and executes the program in all possible ways to detect errors. Although JPF cannot read files compiled by the HJLib runtime, there is a Verification Runtime (VR) specially developed to run HJ programs in JPF. JPF can be used output information enough to create computations graphs for the HJ programs.

